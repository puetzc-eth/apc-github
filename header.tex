\usepackage[english]{babel}
\usepackage{pstricks}
\usepackage{graphicx}
\usepackage{supertabular}
\usepackage{caption}
\captionsetup{format=plain}
\usepackage{amsmath,amssymb,latexsym,mathtools} %Mathe
\usepackage{isomath}                            %nicht-kursive griechische Buchstaben
\usepackage{pifont}
\usepackage{caption, booktabs}
\usepackage{float}
\usepackage{extarrows}                          %Pfeile in den Formeln
\usepackage{siunitx}                            %SI-Einheiten
\usepackage{color, colortbl}                              %Farbige Texte
\usepackage{pdfpages}                           %PDFs einfügen für den Anhang
\usepackage{listings}                           %Aufzählungen
\usepackage{chemmacros}                         %--Allgemeines Chemiepackage--
\usepackage[version=4]{mhchem}                  %Chemische Reaktionsgleichungen
\usepackage{tikz, pgfplots}   
\usepackage{chemfig}                            %Chemische Strukturformeln
\usepackage{hpstatement}                        %P-H-Sätze
\usepackage{tabularx}
\usepackage{framed}  %PDFs mit Rahmen
\usepackage{xcolor}
\usepackage{mdframed}
\usepackage{cancel}
\usepackage{epstopdf}
\usepackage{longtable}
\usepackage{alphabeta}
\usepackage{esint}
\usepackage{bbm}
\usepackage{textgreek}
\usepackage{xcolor}
\usepackage{hyperref}    
\usepackage{textcomp}
\hypersetup{
    colorlinks=true,
    linkcolor=red,
    urlcolor=black,
    pdftitle={document}
}

\definecolor{codegreen}{rgb}{0,0.6,0}
\definecolor{codegray}{rgb}{0.5,0.5,0.5}
\definecolor{codepurple}{rgb}{0.58,0,0.82}
\definecolor{backcolour}{rgb}{0.95,0.95,0.92}

%colorstyle coding
\lstdefinestyle{mystyle}{
    backgroundcolor=\color{backcolour},   
    commentstyle=\color{codegreen},
    keywordstyle=\color{magenta},
    numberstyle=\tiny\color{codegray},
    stringstyle=\color{codepurple},
    basicstyle=\ttfamily\footnotesize,
    breakatwhitespace=false,         
    breaklines=true,                 
    captionpos=b,                    
    keepspaces=true,                 
    numbers=left,                    
    numbersep=5pt,                  
    showspaces=false,                
    showstringspaces=false,
    showtabs=false,                  
    tabsize=2
}

%Cpp style from VS code

\definecolor{ce_yellow}{rgb}{0.902,0.859,0.455}
\definecolor{ce_gray}{rgb}{0.459,0.443,0.369}
\definecolor{ce_lime}{rgb}{0.459,0.816,0.180}
\definecolor{ce_pink}{rgb}{0.976,0.149,0.447}
\definecolor{ce_cyan}{rgb}{0.40,0.851,0.937}
\definecolor{ce_violet}{rgb}{0.545,0.506,1.00}
\definecolor{ce_back}{rgb}{0.153,0.157,0.133}
\definecolor{ce_white}{rgb}{1,1,1}

\lstdefinestyle{CodeExpert}{
    language=C++,
    basicstyle=\ttfamily\linespread{0.8}\color{ce_white},
    numbers=none,
    aboveskip=0mm,
    belowskip=0mm,
    frame = none,
    numberstyle=\tiny\color{ce_grey},
    backgroundcolor = \color{ce_back},
    keywordstyle=\color{ce_cyan},
    commentstyle=\color{ce_gray},
    stringstyle=\color{ce_yellow},
    morecomment=[n][\color{ce_pink}]{\#}{\ },
    literate=
    *{./}{{{\color{ce_pink}./}}}2
    {.^}{{{\color{ce_pink}.\^{}}}}2
    {=}{{{\color{ce_pink}=}}}1
    {+}{{{\color{ce_pink}+}}}1
    {*}{{{\color{ce_pink}*}}}1
    {-}{{{\color{ce_pink}-}}}1
    {&}{{{\color{ce_pink}&}}}1
    {<<}{{{\color{ce_pink}<<}}}2
    {>>}{{{\color{ce_pink}>>}}}2
    {<}{{{\color{ce_pink}<}}}1
    {>}{{{\color{ce_pink}>}}}1
    {->}{{{\color{ce_pink}->}}}2
    {1}{{{\color{ce_violet}1}}}1
    {2}{{{\color{ce_violet}2}}}1
    {3}{{{\color{ce_violet}3}}}1
    {4}{{{\color{ce_violet}4}}}1
    {5}{{{\color{ce_violet}5}}}1
    {6}{{{\color{ce_violet}6}}}1
    {7}{{{\color{ce_violet}7}}}1
    {8}{{{\color{ce_violet}8}}}1
    {9}{{{\color{ce_violet}9}}}1
    {0}{{{\color{ce_violet}0}}}1
    {this}{{{\color{ce_lime}this}}}1
    {if}{{{\color{ce_pink}if}}}1
    {do}{{{\color{ce_pink}do}}}1
    {for}{{{\color{ce_pink}for}}}1
    {else}{{{\color{ce_pink}else}}}1
    {then}{{{\color{ce_pink}then}}}1
    {break}{{{\color{ce_pink}break}}}1
    {continue}{{{\color{ce_pink}continue}}}1
    {public}{{{\color{ce_pink}public}}}1
    {private}{{{\color{ce_pink}private}}}1
    {while}{{{\color{ce_pink}while}}}1
    {continue}{{{\color{ce_pink}continue}}}1
    {nullptr}{{{\color{ce_violet}nullptr}}}1
    {NULL}{{{\color{ce_violet}NULL}}}1,
}

\definecolor{cp_codekeyword}{rgb}{1,0,0.2}
\definecolor{cp_codecomment}{rgb}{0.6,0.6,0.6}
\definecolor{cp_backcolour}{rgb}{0.2,0.2,0.2}
%\definecolor{cp_backcolour}{rgb}{0.95,0.95,0.92}
\definecolor{cp_code}{rgb}{1,1,1}

\definecolor{cp_codegreen}{rgb}{0,0.6,0}
\definecolor{cp_codegray}{rgb}{0.5,0.5,0.5}
\definecolor{cp_codepurple}{rgb}{0.58,0,0.82}


\lstdefinestyle{cp_cpp}{
    language=C++,
    backgroundcolor=\color{cp_backcolour},   
    commentstyle=\color{cp_codecomment},
    keywordstyle=\color{cp_codekeyword},
    numberstyle=\tiny\color{cp_codegray},
    stringstyle=\color{cp_codepurple},
    basicstyle=\ttfamily\footnotesize\color{cp_code},
    breakatwhitespace=false, 
    breaklines=false,                 
    captionpos=b,
    keepspaces=false,     %            
    numbers=left,                    
    numbersep=5pt,                  
    showspaces=false,                
    showstringspaces=false,
    showtabs=false,                  
    tabsize=2
}

%\lstset{basicstyle=\ttfamily,frame=tb,aboveskip=1mm,belowskip=1mm,showstringspaces=true,columns=flexible,breaklines=true,breakatwhitespace=true,tabsize=2,}

\lstset{style=mystyle}

%\usepackage{fancyhdr}

%\usepackage[backend=biber,style=chem-acs]{biblatex}     %Literatur
%\addbibresource{LitGdM1.bib} 

\usepackage[headsepline]{scrlayer-scrpage}
\pagestyle{scrheadings}
\clearscrheadfoot
\ohead{\pagemark}
\ihead{\headmark}
\automark[section]{section}

\setlength{\parindent}{0em}
\setlength{\parskip}{1ex}
%\pagestyle{headings}
\newcommand{\MyOwnCaption}{\captionsetup{font=small}}   %Befehl für Schriftgrösse von Bildbeschreibungen
\setlength{\textwidth}{16cm}
\setlength{\textheight}{23cm}
\setlength{\evensidemargin}{0.0cm}
\setlength{\oddsidemargin}{0.0cm}
\numberwithin{equation}{section}                     %Gleichungen nummerieren

%Neue Befehle:
\newcommand{\dede}[2]{\frac{\mathrm{d}#1}{\mathrm{d}#2}}
\newcommand{\parpar}[2]{\frac{\partial #1}{\partial #2}}
\newcommand{\de}{\mathrm{d}}
\newcommand{\Psirm}{\mathrm{\Psi}}
\newcommand{\Deltarm}{\mathrm{\Delta}}
\newcommand{\deff}[2]{\textbf{#1: }\textit{#2}}
\newcommand{\expp}[1]{\exp{\left\{ #1 \right\}}}
\newcommand{\sinn}[1]{\sin{\left( #1 \right)}}
\newcommand{\coss}[1]{\cos{\left( #1 \right)}}
\newcommand{\lnn}[1]{\ln{\left\{ #1 \right\}}}
\newcommand{\shortref}[4]{\begin{flushright}[\textit{#1}\;\textbf{#2},\;\textit{#3},\;#4]\end{flushright}}
\newcommand{\mdred}[2]{\begin{mdframed}[backgroundcolor=red!20]\textbf{#1} #2 \end{mdframed}}
\newcommand{\mdblue}[2]{\begin{mdframed}[backgroundcolor=blue!20]\textbf{#1} #2 \end{mdframed}}
\newcommand{\mdgreen}[2]{\begin{mdframed}[backgroundcolor=green!20]\textbf{#1} #2 \end{mdframed}}
\newcommand{\kb}{k_\mathrm{B}}
\newcommand{\figapp}[3]{\begin{figure}[H]
    \centering
    \MyOwnCaption\psset{unit=1cm}
       \centering
       \includegraphics[height=\linewidth,angle=90]{#1}
    \parbox[b]{15.0cm}{
\caption{#2}\label{#3}}
\end{figure}}

%---Make Title---
\newcommand{\titlee}[2]{\thispagestyle{empty}\begin{center}\begin{LARGE}\sectfont #1 \end{LARGE}\\\vspace{0.5cm} Connor Pütz \textit{puetzc@ethz.ch} (BSc Chemie -- #2)\\\vspace{1cm}\end{center}\tableofcontents\newpage}

\newcommand{\titleee}[2]{\thispagestyle{empty}\begin{center}\begin{LARGE}\sectfont #1 \end{LARGE}\\\vspace{0.5cm} Connor Pütz \textit{puetzc@ethz.ch} (BSc Chemie ETH Zürich -- #2)\\\vspace{1cm}\end{center}\tableofcontents\newpage}


\newcommand{\intt}[4]{\int\limits_{#1}^{#2}#3 \ \de #4}
\newcommand{\vecc}[3]{\begin{pmatrix} #1 \\ #2 \\ #3 \end{pmatrix}}

\newcommand{\shortRefUrl}[5]{\begin{flushright}\href{ #5 }{[\textit{#1}\;\textbf{#2},\;\textit{#3},\;#4]}\end{flushright}}

\DeclareCaptionLabelFormat{custom}{\textbf{#1 #2}}
\DeclareCaptionLabelSeparator{custom}{:}    
\DeclareCaptionFormat{custom}{#1 #2 #3}
\captionsetup{format=custom,labelformat=custom,labelsep=custom}

\usepackage{comment}

\usepackage[most]{tcolorbox}
%\usepackage{afterpage}
\usepackage{lmodern}

\newcommand{\tred}[2]{\begin{tcolorbox}[colback=red!5!white, colframe=red!50!black, title=#1, breakable] #2 \end{tcolorbox}}

\newcommand{\tblue}[2]{\begin{tcolorbox}[colback=blue!5!white, colframe=blue!50!black, title=#1, breakable] #2 \end{tcolorbox}}

\newcommand{\tgreen}[2]{\begin{tcolorbox}[colback=green!5!white, colframe=green!50!black, title=#1, breakable] #2 \end{tcolorbox}}

\newcommand{\tyellow}[2]{\begin{tcolorbox}[colback=yellow!5!white, colframe=yellow!50!black, title=#1, breakable] #2 \end{tcolorbox}}

\newcommand{\e}[1]{\mathrm{e}^{ #1 }}
\newcommand{\ima}{\mathrm{i}}

%tikz

\usetikzlibrary{positioning}

\usepackage{xfrac}

\usepackage{enumitem}