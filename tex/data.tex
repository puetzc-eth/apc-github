% Data structures (week 2-3)

\deff{Data Type}{A data type is a classification for one of several types of data. A distinction is made between primitive data types (int double, float, bool, char) and advanced data types such as classes or heaps.}

\begin{itemize}
    \item \textbf{Composite data types:} contain multiple primitive data types
    \item \textbf{Classes:} contain primitive data types and functions
    \item \textbf{Abstract data types:} type that does not specify an implementation
\end{itemize}

\deff{Data Structure}{A data structure is a combination of several identical or different data types.}

\subsection{Composite Data Types}

\subsubsection{Structures}

\deff{Structures}{Structures is a data type whose variables and functions are always public. Normally structs are only used for data storage, as otherwise classes are used whose members are private by default.}

Since the members of a structs are public, the members of a structs can also be accessed without getter or setter functions (with a dot between struct name and variable name).

\lstinputlisting[language=C++]{src/data/struct.cpp}

\subsubsection{Unions}

\deff{Unions}{}

\subsubsection{Enumerations}

\subsection{Classes}

\deff{Class}{}

\lstinputlisting[language=C++]{src/data/class.cpp}

\subsection{Abstract Data Types}

\subsubsection{Stacks and queues}

\subsubsection{Linked lists}

\subsubsection{Trees (binary search tree)}

\subsubsection{Heaps}

\subsubsection{Hash tables}

\subsection{Contiguous implementation}

\subsection{Pointers}

\subsection{Dynamic implementation}

\subsection{Standard Template Library (STL)}

\subsection{Hash table example}