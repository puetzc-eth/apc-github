% Cheminformatics (4 weeks)

Chemical informatics is the branch of chemistry that attempts to solve chemical problems algorithmically on the computer, such as predicting reactivity based on substructure search or quantum chemical DFT calculations. \emph{RDKit} has become particularly established for this. It is an open-source based cheminformatics toolkit written in C++, but can also be used with Python, Java or JavaScript. Among other things, it includes the following functionalities:

\begin{itemize}
    \item Reading and writing of molecules
    \item Working with molecules in 2D and 3D
    \item Drawing 2D depictions
    \item Substructure search
    \item Chemical transformations
    \item Maximum common substructure
    \item Fingerprints and molecular similarity
    \item Descriptor calculation
    \item Chemical reactions
    \item Chemical features and pharmacophores
\end{itemize}

A central and fundamental area of cheminformatics is the digital \emph{representation of molecules}. For this, we had already introduced graph theory, whereby a molecule can be represented as a graph (vertices are the atoms and edges and their weight are the bonds as well as their bond multiplicity) using an adjacency list or adjacency matrix. Additional information, such as element, stereochemistry, charge or aromaticity, can also be stored in the vertex. The problem with this representation is that it is not very efficient for storage and molecules are difficult to compare. The standard file format for saving chemical structures with coordinates (from crystal structures) is, however, the \emph{Molefile} format, which works with a connection table and is shown below.

\begin{center}\includegraphics[width=0.85\textwidth]{img/cheminformatics/DataFormat.png}\end{center}

\subsection{1D representation}

The most efficient way to represent molecules for storage is a 1D representation. Furthermore, it can be easily interpreted by a computer and searched in a database. However, it is important that the representation is \emph{reversible}, that the transformations 2D $\rightarrow$ 1D and 1D $\rightarrow$ 2D can be carried out without any problems, and that the representation is \emph{unique}. To achieve this, stereochemistry and aromaticity must be retained in the representation. Examples of such representations are the IUPAC name, WLN, SLN, SMILES and InChI, although we will only deal with the latter two in the following.

\subsubsection{SMILES}

SMILES stands for \emph{Simplified Molecular Input Line Entry System}. It was introduced in 1988 and is based on representing the chemical structure using letters according to established rules. On the computer, a \emph{minimum spanning tree} is first created from a graph, which is then translated into the SMILES using a \emph{depth-first} algorithm, whereby different applicable SMILES are generated depending on which atom is started with. Therefore, a \emph{canonicalization} is needed to make the representation unique. SMILES follows the following rules:

\begin{itemize}
    \item Hydrogens as well as single and aromatic bonds are usually omitted, but can be specified explicitly if desired.
    \item \textbf{Atoms:}
    \begin{itemize}
        \item General: Atomic symbol in square brackets
        \item “Organic” subset (= B, C, N, O, P, S, F, Cl, Br, I) can be written without brackets if the number of attached Hs is “normal”.
        \item Attached hydrogens and formal charges always specified inside brackets.
        \item Atoms in aromatic rings are specified by lower case letters (i.e. c, n).
        \item Stereocentres are specified with @ (anti-clockwise writing of neighbors) and @@ (clockwise writing of neighbors) inside brackets.
    \end{itemize}
    \begin{center}\includegraphics[width=0.85\textwidth]{img/cheminformatics/SmilesRulesAtoms.png}\end{center}
    \item \textbf{Bonds:}
    \begin{itemize}
        \item Single bond: “-”, double bond: “=“, triple bond: “#”, aromatic bond: “:”
        \item Cis/trans double bonds: use of “/” and “\”, e.g.
    \end{itemize}
    \begin{center}\includegraphics[width=0.85\textwidth]{img/cheminformatics/SmilesRulesBonds.png}\end{center}
    \item \textbf{Branches:}
    \begin{itemize}
        \item Specified by parentheses
        \item Can be nested or stacked
    \end{itemize}
    \begin{center}\includegraphics[width=0.85\textwidth]{img/cheminformatics/SmilesRulesBranches.png}\end{center}
    \item \textbf{Cyclic structures:}
    \begin{itemize}
        \item Represented by breaking one bond (to get spanning tree) and numbering the ring-closure atoms
        \item Ring-closure digits can be reused
    \end{itemize}
    \begin{center}\includegraphics[width=0.85\textwidth]{img/cheminformatics/SmilesRulesCyclic.png}\end{center}
    \item \textbf{Aromaticity}
    \begin{itemize}
        \item Aromaticity in cheminformatics is a concept! Different definitions/algorithms exist (discussed later). Do not confuse it with a physical phenomenon
        \item Aromatic bonds are usually omitted
        \item Atoms in an aromatic ring are specified by lower case letters
    \end{itemize}
    \begin{center}\includegraphics[width=0.85\textwidth]{img/cheminformatics/SmilesRulesAromaticity.png}\end{center}
\end{itemize}

\subsubsection{Canonicalization}

\subsubsection{InChI}

\subsubsection{Ring perception}

\subsubsection{Aromaticity detection}

\subsection{Substructure search}

\subsubsection{SMARTS}

\subsubsection{Subgraph isomorphism}

\subsection{Chemical reactions}

\subsection{Dimensionality reduction}

\subsection{Fingerprints}

\subsection{Maximum common substructure}

\subsection{Scaffolds}

\subsection{Generation of 3D coordinates}

\subsubsection{Distance geometry}

\subsection{Clustering}

\subsubsection{Hierarchical}

\subsubsection{Application to chemical space}

\subsubsection{Non-hierarchical}

\subsubsection{Application to conformations}