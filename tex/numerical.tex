\subsection{Systems of Equations}

\subsubsection{Gauss-elimination}

\textit{Given a Matrix of $\mathrm{N^2}$ dimensions, the Gauss-elimination algorithm will perform a classical gauss elimination as done on paper. For a given Matrix A, and its corresponding coefficient vector b, it will stepwise apply equation \ref{eq: Gauss_elim}. Keep in mind, that the actual algorithm will not have a separate b vector, but instead one largeer A matrix with N rows and N+1 columns}

\begin{align}
    A[j][k] = A[j][k] - A[i][k] \cdot \frac{A[j][i]}{A[i][i]}
    \label{eq: Gauss_elim}
\end{align}

\begin{align*}
    \begin{bmatrix}
        5 & 2 & 1 \\
        2 & 2 & 7 \\
        6 & 2 & 3 \\
    \end{bmatrix}
    =
    \begin{bmatrix}
        5 \\
        3 \\
        3 \\
    \end{bmatrix}
\end{align*}

\textit{The algorithm loops over all rows i and sets element A[j][k] to the new value. This is done by subtracting a[i][k] (so the value of the same column, one row above) with the adjusting factor $\mathrm{\frac{A[j][i]}{A[i][i]}}$, which is the first non-zero element of row j, divided by the first non-zero element of i, which is also the diagonal element. Note that k (the column iterator) starts at the right, as it would otherwise overwrite the first element of the j row, which would become zero, giving the false result for the other elements.}

\textit{For the $\mathrm{2^{nd}}$ row, the algorithm will apply equation \ref{eq: Gauss_elim} as follows}

\begin{align*}
    \begin{bmatrix}
        5 & 2 & 1 \\
        \color{red}2 & \color{red}2 & \color{red}7 \\
        6 & 2 & 3 \\
    \end{bmatrix}
    =
    \begin{bmatrix}
        5 \\
        \color{red}3 \\
        3 \\
    \end{bmatrix}
\end{align*}

\begin{align*}
    A[2][1] &= 2 - 5 \cdot \frac{2}{5} = 0 \\
    A[2][2] &= 2 - 2 \cdot \frac{2}{5} = 1.2 \\
    A[2][3] &= 7 - 1 \cdot \frac{2}{5} = 6.6 \\ \\
    b[2] &= 3 - 5 \cdot \frac{2}{5} = 1
\end{align*}

\begin{align*}
    \begin{bmatrix}
        5 & 2 & 1 \\
        \color{red}0 & \color{red}1.2 & \color{red}6.6 \\
        6 & 2 & 3 \\
    \end{bmatrix}
    =
    \begin{bmatrix}
        5 \\
        \color{red}1 \\
        3 \\
    \end{bmatrix}
\end{align*}

\textit{Analogously for the $\mathrm{3^{rd}}$ row.}

\begin{align*}
    \begin{bmatrix}
        5 & 2 & 1 \\
        0 & 1.2 & 6.6 \\
        \color{red}0 & \color{red}-0.4 & \color{red}1.8 \\
    \end{bmatrix}
    =
    \begin{bmatrix}
        5 \\
        1 \\
        \color{red}-3 \\
    \end{bmatrix}
\end{align*}

\textit{The algorithm now divides the first row by its first element.}

\begin{align*}
    \begin{bmatrix}
        \color{red}1 & \color{red}0.4 & \color{red}0.2 \\
        0 & 1.2 & 6.6 \\
        0 & -0.4 & 1.8 \\
    \end{bmatrix}
    =
    \begin{bmatrix}
        \color{red}1 \\
        1 \\
        -3 \\
    \end{bmatrix}
\end{align*}

\textit{Now, i is increased by one, and the same process is repeated to eliminate $\mathrm{A[3][2] = -0.4}$.}

\begin{align*}
    \begin{bmatrix}
        1 & 0.4 & 0.2 \\
        0 & 1.2 & 6.6 \\
        0 & \color{red}0 & \color{red}4 \\
    \end{bmatrix}
    =
    \begin{bmatrix}
        1 \\
        1 \\
        \color{red}-2.67 \\
    \end{bmatrix}
\end{align*}

\textit{The algorithm will now again divide the $\mathrm{2^{nd}}$ row by its first non-zero, and iterate to the next one. In this case, its the last row, where $\mathrm{i+1 = j \geq N}$, which skips the loop and goes straight to dividing by the first non-zero element (in this case 4)}

\begin{align*}
    \begin{bmatrix}
        1 & 0.4 & 0.2 \\
        0 & 1 & 5.5 \\
        0 & 0 & 1 \\
    \end{bmatrix}
    =
    \begin{bmatrix}
        1 \\
        0.83 \\
        -0.67 \\
    \end{bmatrix}
\end{align*}

\textit{For the substitution, we now use equation \ref{eq: Gauss_subst}. Note, that A[j][N-1] is the corresponding row in the b vector. In this case, x is a vector that has been initialized to 0.}

\begin{align}
    \label{eq: Gauss_subst}
    A[j][N-1] = A[j][N-1] - A[j][k] * x[k]
\end{align}

\begin{align*}
    \begin{bmatrix}
        1 & 0.4 & 0.2 \\
        0 & 1 & 5.5 \\
        0 & 0 & 1 \\
    \end{bmatrix}
    =
    \begin{bmatrix}
        1 \\
        0.83 \\
        -0.67 \\
    \end{bmatrix}
    \rightarrow
    x =
    \begin{bmatrix}
        0 \\
        0 \\
        0 \\
    \end{bmatrix}
\end{align*}

\textit{This is trivial for the first row, as it is already solved and has only one variable. Subsequently, $\mathrm{x_j}$ is set to $\mathrm{\frac{a[j][N-1]}{a[j][j]}}$ where a[j][j] is the diagonal element. For the $\mathrm{3^{rd}}$ row, this looks as such.}

\begin{align*}
    \begin{bmatrix}
        1 & 0.4 & 0.2 \\
        0 & 1 & 5.5 \\
        0 & 0 & 1 \\
    \end{bmatrix}
    =
    \begin{bmatrix}
        1 \\
        0.83 \\
        -0.67 \\
    \end{bmatrix}
    \rightarrow
    x =
    \begin{bmatrix}
        0 \\
        0 \\
        \color{red}-0.67 \\
    \end{bmatrix}
\end{align*}

\textit{The next row is slightly more sophisticated. Using equation \ref{eq: Gauss_subst} yields:}

\begin{align*}
    b[2] = 0.83 - 5.5 \cdot -0.67 = 4.5\\
\end{align*}

\begin{align*}
    \begin{bmatrix}
        1 & 0.4 & 0.2 \\
        0 & 1 & 0 \\
        0 & 0 & 1 \\
    \end{bmatrix}
    =
    \begin{bmatrix}
        1 \\
        \color{red}4.5 \\
        -0.67 \\
    \end{bmatrix}
    \rightarrow
    x =
    \begin{bmatrix}
        0 \\
        \color{red}4.5 \\
        -0.67 \\
    \end{bmatrix}
\end{align*}

\textit{And lastly:}

\begin{align*}
    b[1^\ast] &= 1 - 0.2 \cdot -0.67 = 1.13 \\
    b[1] &= 1.33 - 0.4 \cdot 4.5 = -0.67 \\
\end{align*}

\begin{align*}
    \begin{bmatrix}
        1 & 0 & 0 \\
        0 & 1 & 5.5 \\
        0 & 0 & 1 \\
    \end{bmatrix}
    =
    \begin{bmatrix}
        \color{red}-0.67 \\
        4.5 \\
        -0.67 \\
    \end{bmatrix}
    \rightarrow
    x =
    \begin{bmatrix}
        \color{red}-0.67 \\
        4.5 \\
        -0.67 \\
    \end{bmatrix}
\end{align*}